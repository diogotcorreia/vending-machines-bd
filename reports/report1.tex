\documentclass[12pt,a4paper]{article}
\usepackage[legalpaper, portrait, margin=3cm]{geometry}
\usepackage{fancyhdr}
\usepackage{amsmath}
\usepackage{amssymb}
\usepackage{graphicx}
\usepackage{wrapfig}
\usepackage{blindtext}
\usepackage{hyperref}
\usepackage{pdflscape}
\usepackage{svg}

\graphicspath{ {./} }
\hypersetup{
  colorlinks=true,
  linkcolor=blue,
  filecolor=magenta,
  urlcolor=blue,
  citecolor=blue,
  pdftitle={Relatório BD - Entrega 1 - 2021/2022},
  pdfpagemode=FullScreen,
}

\pagestyle{fancy}
\fancyhf{}
\rhead{Grupo \textbf{8}}
\lhead{Relatório Entrega 1 BD 2021/2022 LEIC-A}
\cfoot{Diogo Gaspar (99207), Diogo Correia (99211) e Tomás Esteves (99341)}

\renewcommand{\footrulewidth}{0.2pt}

\renewcommand{\labelitemii}{$\circ$}
\renewcommand{\labelitemiii}{$\diamond$}


\begin{document}
  \begin{titlepage}
    \begin{center}
      \vspace*{5cm}
    
      \Huge
      \textbf{Projeto BD - Parte 1}

      \vspace{0.5cm}
      \LARGE
      Grupo 008 | Turno L20 | LEIC-A

      \vspace{0.5cm}
      \large
      Prof. João Aparício | Prof. Leonardo Alexandre

      \vfill
    \end{center}
    \large
    \begin{itemize}
      \item[] \textbf{Diogo Gaspar} (99207) - 34\% - Xh
      \item[] \textbf{Diogo Correia} (99211) - 33\% - Xh
      \item[] \textbf{Tomás Esteves} (99341) - 33\% - Xh
    \end{itemize}
  \end{titlepage}

  \begin{landscape}
    \includesvg[width=1.6\textwidth]{er_model}
  \end{landscape}

  \section*{Restrições de Integridade}
  \begin{itemize}
    \item[\textbf{(IC-1)}] Every shelf where the category is presentable
      \textbf{must have} the same temperature.
    \item // TODO
  \end{itemize}

  \section*{Justificações de Desenho}
  \begin{itemize}
    \item Interpretámos \textit{"Os produtos [...] têm [...] marca ou nome"}
      como não exclusivo, ou seja, permitir que um produto tenha \textbf{apenas marca},
      \textbf{apenas nome} ou ainda \textbf{marca e nome}, mas sem nunca
      permitir que o produto não tenha nenhum deles.
    \item Não incluímos no diagrama nada relativo a
      \textit{"O sistema deve determinar, para uma super categoria, quantas sub categorias existem."},
      visto que se trata de uma propriedade que é calculável através de
      atributos já existentes na entidade \textbf{Category}.
    \item Considerámos que a frase
      \textit{"O pessoal reabastecedor deve assegurar que a prateleira esteja limpa e organizada."}
      não constitui um requisito de desenho mas sim um requisito de funcionalidade,
      pelo que não foi incluido no diagrama.
    \item Considerámos que um evento de reabastecimento (\textbf{Restock Event}) é
      identificado pelo momento em que ocorre (\textbf{timestamp}) e pela associação
      prateleira/produto (\textbf{Shelf}/\textbf{Product}).
    \item Embora não seja relevante para a correção do modelo, decidimos criar uma
      entidade para a morada (\textbf{Address}) para discriminar os campos que a constituem.
    \item porquê da ligação category <-> shelf
  \end{itemize}
\end{document}
