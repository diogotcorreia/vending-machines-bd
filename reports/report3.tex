\documentclass[12pt,a4paper]{article}
\usepackage[legalpaper, portrait, margin=2cm]{geometry}
\usepackage{fancyhdr}
\usepackage{amsmath}
\usepackage{amssymb}
\usepackage{graphicx}
\usepackage{wrapfig}
\usepackage{blindtext}
\usepackage{hyperref}
\usepackage{enumitem}
\usepackage{pdflscape}
\usepackage{svg}
\usepackage{listings}
\usepackage{xcolor}

\graphicspath{ {./} }
\hypersetup{
  colorlinks=true,
  linkcolor=blue,
  filecolor=magenta,
  urlcolor=blue,
  citecolor=blue,
  pdftitle={Relatório BD - Entrega 3 - 2021/2022},
  pdfpagemode=FullScreen,
}

\pagestyle{fancy}
\fancyhf{}
\rhead{Grupo \textbf{8}}
\lhead{Relatório Entrega 3 BD 2021/2022 LEIC-A}
\cfoot{Diogo Gaspar (99207), Diogo Correia (99211) e Tomás Esteves (99341)}

\renewcommand{\footrulewidth}{0.2pt}

\renewcommand{\labelitemii}{$\circ$}
\renewcommand{\labelitemiii}{$\diamond$}
\newcommand{\op}{\text}

\newlist{constraintsList}{itemize}{4}
\setlist[constraintsList]{itemsep=1pt, topsep=1pt, label=\protect\mpbullet}

\definecolor{codegreen}{rgb}{0,0.6,0}
\definecolor{codegray}{rgb}{0.5,0.5,0.5}
\definecolor{codepurple}{rgb}{0.58,0,0.82}
\definecolor{backcolour}{rgb}{0.95,0.95,0.92}

\lstdefinestyle{mystyle}{
    commentstyle=\color{codegreen},
    keywordstyle=\color{magenta},
    numberstyle=\tiny\color{codegray},
    stringstyle=\color{codepurple},
    basicstyle=\ttfamily\footnotesize,
    breakatwhitespace=false,
    breaklines=true,
    captionpos=b,
    keepspaces=true,
    numbers=left,
    numbersep=5pt,
    showspaces=false,
    showstringspaces=false,
    showtabs=false,
    tabsize=2
}

\lstset{style=mystyle}
% https://stackoverflow.com/questions/1116266/listings-in-latex-with-utf-8-or-at-least-german-umlauts
\lstset{%
        inputencoding=utf8,
        extendedchars=true,
        literate=%
        {é}{{\'{e}}}1
        {è}{{\`{e}}}1
        {ê}{{\^{e}}}1
        {ë}{{\¨{e}}}1
        {É}{{\'{E}}}1
        {Ê}{{\^{E}}}1
        {û}{{\^{u}}}1
        {ù}{{\`{u}}}1
        {ú}{{\'{u}}}1
        {â}{{\^{a}}}1
        {à}{{\`{a}}}1
        {á}{{\'{a}}}1
        {ã}{{\~{a}}}1
        {Á}{{\'{A}}}1
        {Â}{{\^{A}}}1
        {Ã}{{\~{A}}}1
        {ç}{{\c{c}}}1
        {Ç}{{\c{C}}}1
        {õ}{{\~{o}}}1
        {ó}{{\'{o}}}1
        {ô}{{\^{o}}}1
        {Õ}{{\~{O}}}1
        {Ó}{{\'{O}}}1
        {Ô}{{\^{O}}}1
        {î}{{\^{i}}}1
        {Î}{{\^{I}}}1
        {í}{{\'{i}}}1
        {Í}{{\~{Í}}}1
}

\begin{document}
  \begin{titlepage}
    \begin{center}
      \vspace*{5cm}

      \Huge
      \textbf{Projeto BD - Parte 3}

      \vspace{0.5cm}
      \LARGE
      Grupo 008 | Turno L20 | LEIC-A

      \vspace{0.5cm}
      \large
      Prof. João Aparício | Prof. Leonardo Alexandre

      \vfill
    \end{center}
    \large
    \begin{itemize}
      \item[] \textbf{Diogo Gaspar} (99207) - 33\% - Xh
      \item[] \textbf{Diogo Correia} (99211) - 34\% - Xh
      \item[] \textbf{Tomás Esteves} (99341) - 33\% - Xh
    \end{itemize}
  \end{titlepage}

  \section*{Restrições de Integridade}

  \begin{enumerate}
    \item \textbf{(RI-1) Uma Categoria não pode estar contida em si própria.}

    Considerámos que a noção de "contida em si própria" se estendia para além de uma
    relação direta, não podendo, assim, haver quaisquer ciclos na hierarquia da categoria.

    \lstinputlisting[language=SQL]{sql/report3/integrity-constraints/ic1.sql}

    \item \textbf{(RI-4) O número de unidades repostas num Evento de Reposição
    não pode exceder o número de unidades especificado no Planograma.}

    \lstinputlisting[language=SQL]{sql/report3/integrity-constraints/ic2.sql}

    \item \textbf{(RI-5) Um Produto só pode ser reposto numa Prateleira que apresente
    (pelo menos) uma das Categorias desse produto.}

    \lstinputlisting[language=SQL]{sql/report3/integrity-constraints/ic3.sql}
  \end{enumerate}

  De realçar que todas as restrições de integridade descritas acima estão também
  presentes em \texttt{ICs.sql}.
  
  \section*{SQL}

  \begin{enumerate}
    \item \textbf{Qual o nome do retalhista (ou retalhistas) responsáveis pela reposição do maior número de categorias?}

    \lstinputlisting[language=SQL]{sql/report3/queries/query1.sql}

    \item \textbf{Qual o nome do ou dos retalhistas que são responsáveis por todas as categorias simples?}

    \lstinputlisting[language=SQL]{sql/report3/queries/query2.sql}

    % otherwise the query would be split in half
    \vspace*{2cm}

    \item \textbf{Quais os produtos (ean) que nunca foram repostos?}

    \lstinputlisting[language=SQL]{sql/report3/queries/query3.sql}

    \item \textbf{Quais os produtos (ean) que foram repostos sempre pelo mesmo retalhista?}

    \lstinputlisting[language=SQL]{sql/report3/queries/query4.sql}
  \end{enumerate}

  De realçar que todas as \textit{queries} descritas acima estão também
  presentes em \texttt{queries.sql}.

  \section*{Vistas}

  % TODO: not sure se é preciso colocar no relatório, either way fica aqui o placeholder

  Pretendia-se que criássemos a vista \texttt{Vendas} (que tomámos a liberdade de renomear
  para \texttt{Sales}), procurando representar todas as saídas de stock de um produto
  (na ausência de desperdícios e roubos). Tendo em conta o enunciado, optámos por criar
  a vista da seguinte forma:

  \lstinputlisting[language=SQL]{sql/report3/views/view.sql}

  Recorremos à função \texttt{EXTRACT}, do \texttt{POSTGRES}, para extrair os campos \texttt{year, 
  quarter, day\_month} e \texttt{day\_week} a partir do campo \texttt{instant} dos
  \texttt{Replenishment Events}, tal como sugerido no enunciado.

  Para consultar a vista na aplicação, o utilizador pode, partindo do menu principal,
  clicar no botão \texttt{Past Sales}, que o redireciona para uma página com uma
  listagem de todas as \texttt{Sales} que ocorreram até então.

  \section*{Aplicação}

  % TODO: same here, não tenho a certeza se precisamos de colocar coisas nesta secção

  Pretendia-se, quanto à aplicação, que esta permitisse ao utilizador realizar um conjunto
  de ações pré-determinado (garantindo a atomicidade de todas elas):

  \begin{enumerate}
    \item \textbf{Inserir e remover Categorias e Sub-Categorias.}
    
    Para \textbf{inserir} ou \textbf{remover} uma \textbf{Categoria} ou \textbf{Sub-Categoria}
    da base de dados, o utilizador, partindo do menu principal, deve selecionar
    uma das opções (\texttt{Category} ou \texttt{Simple Category}), conforme o tipo
    que pretenda inserir ou remover. De seguida:

    \begin{enumerate}
      \item Caso pretenda inserir uma nova \textbf{Categoria} ou \textbf{Sub-Categoria},
      o utilizador deve clicar no botão presente na página respetiva que o permite
      realizar essa ação: \texttt{Insert Simple Category} ou \texttt{Insert Super Category}.
      Será levado para uma outra página, onde poderá inserir o \texttt{name} da nova
      \textbf{Categoria}.

      \item Caso pretenda remover uma \textbf{Categoria} ou \textbf{Sub-Categoria},
      o processo é ligeiramente diferente. É apresentada uma tabela ao utilizador,
      onde cada entrada corresponde a uma \textbf{Categoria}. Deve, então, navegar
      até à entrada correspondente à \textbf{Categoria} que pretende remover e
      clicar no botão \texttt{Remove} que lhe corresponde. 
    \end{enumerate}
    
    \item \textbf{Inserir e remover um Retalhista, com todos os seus produtos.}
    
    \textbf{Inserir} e \textbf{remover} um \textbf{Retalhista} é em tudo semelhante ao processo
    descrito acima para as \textbf{Categorias}: o utilizador deve navegar,
    partindo do menu principal, para a secção \texttt{Retailer}, onde lhe é
    apresentada uma tabela, contendo todos os \textbf{Retalhistas} presentes na
    base de dados, e um botão para adicionar um novo \textbf{Retalhista}:

    \begin{enumerate}
      \item Caso pretenda inserir um novo \textbf{Retalhista}, o utilizador
      deve clicar no botão \texttt{Insert Retailer}, inserindo de seguida os respetivos
      \texttt{name} e \texttt{tin}.

      \item Caso pretenda remover um \textbf{Retalhista}, o utilizador deve, tal
      como nas secções referentes às \textbf{Categorias}, escolher a entrada na
      tabela correspondente ao \textbf{Retalhista} que pretende remover e clicar
      no botão \texttt{Remove} que lhe corresponde.
    \end{enumerate}
    
    \item \textbf{Listar todos os Eventos de Reposição de uma IVM, apresentando
    o número de unidades repostas por Categoria de Produto.}

    O utilizador deve, nesta situação, navegar até à secção \texttt{IVM},
    partindo do menu principal. É-lhe apresentada uma tabela contendo todas as
    \textbf{IVM} presentes na base de dados, e cada entrada tem um botão
    associado, \texttt{List Replenishment Events}, que permite ao utilizador
    visualizar todos os eventos de reposição associados à \textbf{IVM} em questão
    sob a forma de uma tabela.
    
    \item \textbf{Listar todas as Sub-Categorias de uma Super-Categoria, a todos
    os níveis de profundidade.}

    O utilizador deve, aqui, navegar até à secção \texttt{Super Category},
    partindo do menu principal. É-lhe apresentada uma tabela contendo todas as
    \textbf{Super-Categorias} presentes na base de dados, e cada entrada tem um
    botão associado, \texttt{List Sub-Categories}, que permite ao utilizador
    visualizar todas as \textbf{Sub-Categorias} associadas à \textbf{Super-Categoria}
    em questão sob a forma de uma tabela.

  \end{enumerate}

  % TODO: verificar se está tudo atómico e SQL Injection proof

  \section*{Consultas OLAP}

  Pretendia-se que escrevêssemos duas consultas SQL que permitissem analisar o número
  total de artigos vendidos:

  \begin{enumerate}
    \item \textbf{Num dado período (i.e. entre duas datas), por dia da semana, por concelho e no total.}
    
    \lstinputlisting[language=SQL]{sql/report3/olap-queries/olap1.sql}

    \item \textbf{Num dado distrito (i.e. “Lisboa”), por concelho, categoria, dia da semana e no total.}
  
    \lstinputlisting[language=SQL]{sql/report3/olap-queries/olap2.sql}
  \end{enumerate}

  De realçar que todas as \textit{queries} OLAP descritas acima estão também
  presentes em \texttt{analytics.sql}.

  \section*{Índices}

  Foi-nos pedido, ainda, que indicássemos (justificando) os índices que faria sentido
  criar, por forma a \textit{agilizar} a execução de cada uma das \textit{queries} abaixo:

  \begin{enumerate}
    \item \textbf{Primeira \textit{query}:}
    
    \lstinputlisting[language=SQL]{sql/report3/queries/example1.sql}

    \item \textbf{Segunda \textit{query}:}
  
    \lstinputlisting[language=SQL]{sql/report3/queries/example2.sql}
  \end{enumerate}
  
\end{document}
